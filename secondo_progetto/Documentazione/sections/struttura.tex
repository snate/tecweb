\section{Struttura organizzativa}\label{sec:struttura-org}
Dopo aver analizzato e navigato nel sito, è stata individuata la seguente
struttura organizzativa gerarchica:
\begin{figure}[h!]
    \centering
    \scalebox{.7}{\pgfdeclarelayer{bg}    % declare background layer
\pgfsetlayers{bg,main}  % set the order of the layers (main is the standard layer)

\begin{tikzpicture}[
    grow=right,
    level 1/.style={sibling distance=5cm,level distance=5.2cm},
    level 2/.style={sibling distance=3.5cm, level distance=4cm},
    edge from parent/.style={very thick,draw=blue!40!black!60,
        shorten >=5pt, shorten <=5pt},
    edge from parent path={(\tikzparentnode.east) -- (\tikzchildnode.west)},
    kant/.style={text width=2cm, text centered, sloped},
    every node/.style={text ragged, inner sep=2mm},
    punkt/.style={rectangle, rounded corners, shade, top color=white,
    bottom color=blue!50!black!20, draw=blue!40!black!60, very
    thick }
    ]

\node[punkt, text width=5.5em] {\textbf{\textit{Homepage}}}
    %Lower part lv1
    child {
      node[punkt] [rectangle split, rectangle split, rectangle split parts=3,
        text ragged] {
          \textbf{Pagine secondarie}
            \nodepart{two}
              $\text{Chi Siamo}$
            \nodepart{three}
              $\text{Contatti}$
      }
    }
    child {
        node[punkt] [text ragged] (D) {
            \textbf{Accesso}
        }
        child {
          node[punkt] [text ragged] (M) {
                \textbf{Registrazione}
          }
        }
        edge from parent
            node[kant, below, pos=.6] {}
    }
    %RICERCA
    child {
        node[punkt] [rectangle split, rectangle split, rectangle split parts=4,
         text ragged] {
            \textbf{Ricerca}
                  \nodepart{two}
            $\text{Per autore}$
                  \nodepart{three}
            $\text{Per titolo}$
                  \nodepart{four}
            $\text{Per collana}$
        }
        child {
            node [punkt,rectangle split, rectangle split,
            rectangle split parts=2] (R) {
                \textbf{Risultati ricerca}
                  \nodepart{two}
            $\text{...}$
            }
        }
        edge from parent
            node[kant, below, pos=.6] {}
    }
    %Upper part, lv1 GENERI
    child {
        node[punkt][rectangle split, rectangle split, rectangle split parts=7,
         text ragged] (G) {
            \textbf{Generi}
                  \nodepart{two}
            $\text{Fantasy}$
                  \nodepart{three}
            $\text{Horror}$
                  \nodepart{four}
            $\text{Gialli}$
                  \nodepart{five}
            $\text{Fantascienza}$
                  \nodepart{six}
            $\text{Narrativa}$
                  \nodepart{seven}
            $\text{Romanzi Storici}$
        }
        %ELENCO
        child {
            node [punkt,rectangle split, rectangle split,
            rectangle split parts=2] (E) {
                \textbf{Elenco libri}
                  \nodepart{two}
            $\text{...}$
            }
          child {
            node[punkt] [text ragged] (L) {
              \textbf{Pagina libro}
          }
          child {
            node[punkt] [text ragged] (X) {
                \textbf{Aggiunta commento}
            }
          }
          edge from parent
            node[kant, below, pos=.4] {}
    }
  }
};
  \begin{pgfonlayer}{bg}
    \path[every node/.style={font=\sffamily\small}]
    (R) edge[kant, below, pos=.4] (L);
  \end{pgfonlayer}
\end{tikzpicture}
}
    \caption{Organizzazione gerarchica del sito}
    \label{fig:gerarchia}
\end{figure}

Si è deciso di non tracciare l'arco che collega l'homepage alle pagine del
libro per non rendere più difficile la comprensione del grafico.

\subsection{Considerazioni generiche}
Come si può notare, la struttura potrebbe portare a delle navigazioni che si
discostano da una navigazione lineare seguendo la gerarchia del sito.
Ciò sarebbe un buon modo per far arrivare l'utente direttamente al contenuto
ricercato evitando due click in più, ma mancano le indicazioni per poter informare l'utente dove si trova, come detto nella sezione
\ref{sec:user-posiz}.

\subsection{Considerazioni sulla gerarchia}
La gerarchia non è troppo ampia in quanto vi sono meno di dieci opzioni nel
menù principale, non comportando disorientamento cognitivo all'utente.
La profondità della gerarchia è in una situazione di limite,
poichè se, come detto sopra, all'utente interessasse commentare un libro non
presente nella lista in home, avrebbe bisogno di navigare per quattro o cinque
livelli di profondità prima di raggiungere il suo obiettivo.

Seppur il menù principale non sia troppo ampio, questo problema potrebbe
presentarsi nei risultati di ricerche o cercando un libro per genere, dove
facilmente potrebbero presentarsi più di dieci voci.
Tuttavia si ritiene tale problema trascurabile, poichè se un utente arriva in
tali pagine vuol dire che non sa precisamente cosa sta cercando e
apparentemente non vi è un ordine in tali elenchi.

Tuttavia, qualora il gestore del sito notasse che le voci visualizzate
potenzialmente possono raggiungere una soglia minima (e.g. 10), una facile
soluzione sarebbe offrire all'utente la possibilità di ordinare per titolo,
voto utenti, voto LDV, cognome autore, etc. tramite un semplice script
JavaScript.

\subsection{Ipertestualità}
I link che dalla home portano direttamente alla pagina di un libro aggiungono
ipertestualità alla struttura della pagina, aumentandone leggermente la
flessibilità ma non abbastanza da poter soddisfare le esigenze del sito, che
si ipotizza abbia almeno sei bacini di utenza (tanti quanti i generi).

Siccome i due insiemi nella home composti da tre libri non sono mutuamente
esclusivi, è altamente probabile che il più delle volte non vi siano
contemporaneamente sei libri di sei generi diversi nella home.
Dati:
\begin{itemize}
\item tra tutti gli utenti che stanno navigando contemporaneamente sul sito,
ve ne sono almeno sei interessati a generi diversi;
\item uguale probabilità ad ogni genere di libro di apparire in uno dei sei slot nella home;
\end{itemize}
Nel 98,4\% dei casi un utente non troverà un libro di suo interesse nella home.
%TODO scrivere altro?
