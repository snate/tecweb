\section{Esperienza utente}\label{sec:user-exp}
Di seguito vengono riportate le considerazioni riguardo la qualità del sito in
termini di \textbf{user experience}.

\subsection{Consistenza dell'interfaccia}
L'interfaccia è consistente e utilizza sempre gli stessi elementi grafici per
denotare elementi comuni a diverse pagine, eccezion fatta per:
\begin{itemize}
\item i link in \texttt{accesso.html} e \texttt{registrazione.html};
\item la breadcrumb (in tali pagine contrassegnata con l'id \texttt{path}) non
sempre è strutturata nello stesso modo, contenendo una o due voci a seconda
dei casi (e.g. in \texttt{about.html} e nella pagina generata dalla ricerca);
\item l'utente per poter inserire un commento (e quindi completare un task
relativo alla pagina dove è descritto un libro) deve accedere ad una nuova
pagina per scrivere il commento, senza che in questa nuova pagina siano
visualizzate le informazioni sul libro.
\end{itemize}

Sebbene presenti i problemi sopra descritti, l'interfaccia è comunque ritenuta
consistente rispetto al contesto in cui è inserita (una libreria) e alle
medie aspettative che un utente avrebbe visitando un sito di una libreria.

\subsection{Area visibile}
Ogni pagina ha un'ampia intestazione che occupa una buona parte di spazio;
tuttavia, non riducendo la pagina a dimensioni eccessivamente ridotte, il
contenuto principale rimane parzialmente visibile (se non questi, almeno i
titoli del contenuto).

Nelle descrizioni dei libri i commenti occupano una posizione di fondo.
Probabilmente i commenti dovrebbero trovare uno spazio nell'area visibile
(\textit{above the fold}). Se non venissero inseriti in questa zona della
pagina, servirebbe almeno un pulsante o un'ancora presente \textit{above the
fold} per poter raggiungere facilmente la sezione della pagina riservata ai
commenti.

\subsection{Informazioni utili}
L'utente non sempre dispone di informazioni accessorie ma utili durante la sua
navigazione: di seguito verranno discusse la loro presenza e, se presenti, le
modalità con cui queste sono state realizzate.

\subsubsection{Informazioni sulla propria posizione}
L'utente non dispone di sufficienti informazioni sulla barra contestuale
(\textit{path} o \textit{breadcrumb}) per capire in che modo ha raggiunto la
pagina in cui si trova: 

