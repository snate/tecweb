%%%%%%%%%%%%%%%%%%%%%%%%%%%%%%%
%MANCANO ABBREVIAZIONI
% LDV
% J.K. ROWLING
%%%%%%%%%%%%%%%%%%%%%%%%%%%%%%%

%%%%%%%%%%%%%%%%%%%%%%%%%%%%%%%
% ERRORI NEI FORM
% ASSEGNAZIONE CLASSI (STRUTTURA SCORRETTA)
% LABEL MATCHANO NOME MA NON ID
% MANCA SUBMIT
%% ^ANCHE A CAUSA DI CIÒ NON SI PUÒ FARE NULLA DA SCREEN READER O SENZA JS
% FUNZIONANO SOLO VIA JS
% PLACEHOLDER FATTO MALE
% LA RICERCA È EFFETTUATA CON POST E NON CON GET -> NON POSSO SALVARE --
% RISULTATI RICERCA
%%%%%%%%%%%%%%%%%%%%%%%%%%%%%%%

%%%%%%%%%%%%%%%%%%%%%%%%%%%%%%%
% LISTE
% SPESSO SONO USATE UL ANZICHÈ DL
%%%%%%%%%%%%%%%%%%%%%%%%%%%%%%%

%%%%%%%%%%%%%%%%%%%%%%%%%%%%%%%
% META
% MANCANO META TAG UTILI
%%%%%%%%%%%%%%%%%%%%%%%%%%%%%%%

%%%%%%%%%%%%%%%%%%%%%%%%%%%%%%%
% FONT SCELTO
% TALVOLTA POCO LEGGIBILE (E.G. TITOLI)
%%%%%%%%%%%%%%%%%%%%%%%%%%%%%%%

%%%%%%%%%%%%%%%%%%%%%%%%%%%%%%%
% ACCESSO.HTML
% LINK CIRCOLARE IN ALTO A DX
%%%%%%%%%%%%%%%%%%%%%%%%%%%%%%%

%%%%%%%%%%%%%%%%%%%%%%%%%%%%%%%
% EVERYWHERE
% TABINDEX MANCANTI
%%%%%%%%%%%%%%%%%%%%%%%%%%%%%%%

%%%%%%%%%%%%%%%%%%%%%%%%%%%%%%%
% ALT/TITLE
% NON SEMPRE PRESENTI
%%%%%%%%%%%%%%%%%%%%%%%%%%%%%%%

%%%%%%%%%%%%%%%%%%%%%%%%%%%%%%%
% NON VEDO/NON SENTO
% PROVARE LA PAGINA CON VISCHECK
% PROVARE LA PAGINA CON FANGS
%%%%%%%%%%%%%%%%%%%%%%%%%%%%%%%

%%%%%%%%%%%%%%%%%%%%%%%%%%%%%%%
% B/W
% TESTARE LA PAGINA IN BIANCO E NERO
%%%%%%%%%%%%%%%%%%%%%%%%%%%%%%%

%%%%%%%%%%%%%%%%%%%%%%%%%%%%%%%
% TITLE SUI LINK
% MANCANTI DAPPERTUTTO
%%%%%%%%%%%%%%%%%%%%%%%%%%%%%%%

%%%%%%%%%%%%%%%%%%%%%%%%%%%%%%%
% PAROLE STRANIERE
% LA PAROLE STRANIERE NON SONO MARCATE COME TALI
%%%%%%%%%%%%%%%%%%%%%%%%%%%%%%%

%%%%%%%%%%%%%%%%%%%%%%%%%%%%%%%
% SEPARAZIONE STRUTTURA/PRESENTAZIONE
% È PRESENTE UN <hr /> IN contatti.html
% NEL #path C'È UNO SPAN INUTILE INSERITO PROBABILMENTE A SCOPO PRESENTAZIONALE
%%%%%%%%%%%%%%%%%%%%%%%%%%%%%%%

\section{Esperienza utente}\label{sec:user-exp}
Di seguito vengono riportate le considerazioni riguardo la qualità del sito in
termini di \textbf{user experience}.

\subsection{Consistenza dell'interfaccia}
L'interfaccia è consistente e utilizza sempre gli stessi elementi grafici per
denotare elementi comuni a diverse pagine, eccezion fatta per:
\begin{itemize}
\item il blocco di navigazione \textbf{non} presente in tutte le pagine;
\item i link in \texttt{accesso.html} e \texttt{registrazione.html};
\item la breadcrumb (in tali pagine contrassegnata con l'id \texttt{path}) non
sempre è strutturata nello stesso modo, contenendo una o due voci a seconda
dei casi (e.g. in \texttt{about.html} e nella pagina generata dalla ricerca);
\item l'utente per poter inserire un commento (e quindi completare un task
relativo alla pagina dove è descritto un libro) deve accedere ad una nuova
pagina per scrivere il commento, senza che in questa nuova pagina siano
visualizzate le informazioni sul libro.
\end{itemize}

Sebbene presenti i problemi sopra descritti, l'interfaccia è comunque ritenuta
consistente rispetto al contesto in cui è inserita (una libreria) e alle
medie aspettative che un utente avrebbe visitando un sito di una libreria.

\subsection{Area visibile}
Ogni pagina ha un'ampia intestazione che occupa una buona parte di spazio;
tuttavia, non riducendo la pagina a dimensioni eccessivamente ridotte, il
contenuto principale rimane parzialmente visibile (se non questi, almeno i
titoli del contenuto).

Nelle descrizioni dei libri i commenti occupano una posizione di fondo.
Probabilmente i commenti dovrebbero trovare uno spazio nell'area visibile
(\textit{above the fold}). Se non venissero inseriti in questa zona della
pagina, servirebbe almeno un pulsante o un'ancora presente \textit{above the
fold} per poter raggiungere facilmente la sezione della pagina riservata ai
commenti.

\subsection{Informazioni utili}
L'utente non sempre dispone di informazioni accessorie ma utili durante la sua
navigazione: di seguito verranno discusse la loro presenza e, se presenti, le
modalità con cui queste sono state realizzate.

\subsubsection{Informazioni sulla propria posizione}
L'utente se si trova nelle pagine \texttt{accesso.html}, \texttt{logout.html},
\texttt{access_failure.html} o \texttt{registrazione.html} non dispone di un
blocco di navigazione presente nelle altre pagine.

L'utente non dispone di sufficienti informazioni sulla barra contestuale
(\textit{path} o \textit{breadcrumb}) per capire in che modo ha raggiunto la
pagina in cui si trova. Questa è composta solamente da una voce in quasi tutte
le pagine e, dove non è così, sono visualizzati solamente due livelli di
gerarchia e non il completo percorso dalla home fino alla pagina in questione.

I titoli sono brevi e dal particolare al generale; tuttavia presentano i
seguenti vizi di forma:
\begin{enumerate}
\item I titoli delle pagine hanno l'iniziale minuscola e si ritiene che non sia
una scelta effettuata ma un incidente durante la realizzazione delle pagine;
\item La pagina \texttt{accesso.html} ha come titolo ``acceso";
\item La pagina a cui si accede scegliendo un genere letterario ha come titolo
``risultati". Tale titolo non informa sul contenuto presente nella pagina ed è
una forzatura dal punto di vista semantico; oltre a ciò, è un doppione perchè
vi è un'altra pagina (quella dei risultati di una ricerca) che presenta lo
stesso titolo ma offre funzionalità diverse.
\end{enumerate}

\subsubsection{Pagine di aiuto}
Il sito presenta le pagine ``Chi siamo" e ``Contatti", ma mancano:
\begin{itemize}
\item Una pagina dedicata a (ipotetiche) domande frequenti;
\item Una mappa del sito;
\item Una pagina dove viene illustrata la normativa sulla privacy, dal momento
che il sito raccoglie dati sensibili come indirizzi e-mail.
\end{itemize}