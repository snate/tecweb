\section{Accessibilità}
Nel sito non è stato indicato nessuno standard di accessibilità raggiunto.
Tramite un analisi, si è notato inoltre che non vengono soddisfatte nè le
richieste di WAI-A WCAG2.0 nè della ``section 508".
Di seguito verranno infatti descritte le mancanze rilevate nel sito in termini
di accessibilità, ordinate secondo le 14 linee guida di WAI; dove non conforme,
verrà indicato la priorità del punto di controllo dove è stata riscontrata la
mancanza.

\subsection{Linee guida per l'accessibilità}

\subsubsection{Fornire alternative equivalenti al contenuto audio e visivo}
\begin{enumerate}
\item Non sempre viene fornito un equivalente testuale per gli elementi non
testuali, ad esempio nella pagina principale mancano gli attributi \texttt{alt}
ai tag \texttt{img}
(\textit{Priorità 1});
\item Non sono presenti mappe immagine, perciò 1.2 è rispettato;
\item Non sono presenti filmati, perciò 1.3 è rispettato;
\item Non sono presenti filmati, perciò 1.4 è rispettato;
\item Non sono presenti mappe immagine, perciò 1.5 è rispettato;
\end{enumerate}

Il punto di controllo 1.1 fallisce e questo è sufficiente per dire che la
pagina non è accessibile per WAI-A WCAG2.0. Di seguito non verrà più
evidenziato tale fatto (nemmeno per i gradi successivi di accessibilità) poichè
il non-rispetto di un solo punto di controllo compromette il grado complessivo
di accessibilità del sito.

\subsubsection{Non fare affidamento sul solo colore}
\begin{enumerate}
\item L'informazione arriva tramite indicazioni indipendenti dal colore;
\item %CONTRASTO
\end{enumerate}

\subsubsection{Usare marcatori e fogli di stile e farlo in modo appropriato}
\begin{enumerate}
\item Le pagine sono scritte secondo lo standard \textbf{XHTML 1.0 Strict},
sebbene poi non sempre venga rispettato;
\item Tutte le pagine si riferiscono correttamente allo schema \textbf{DTD} di
XHTML 1.0 Strict;
\item Il tag \textit{hr} è utilizzato per meri scopi presentazionali, ovvero
l'inserire un bordo tra delle intestazioni e il contenuto sottostante (ad
esempio in \texttt{about.html}); oltre a questo, sono talvolta inseriti
\texttt{span} per puro scopo presentazionale e non semantico, come quello in
\texttt{\#path} (\textit{Priorità 2});
\item Vengono utilizzate delle unità assolute come \texttt{font-size: medium}
per il testo (\textit{Priorità 2});
\item Gli elementi di intestazione sono usati correttamente, tuttavia si
potrebbe pensare di inserirne altri per favorire la lettura del contenuto da
parte di screen-reader o altri dispositivi che fanno molto affidamento sulle
intestazioni;
\item Vi sono alcune \texttt{dl} che vengono scritte però marcate come
\texttt{ul}, comportando l'aggiunta di tag come \texttt{strong} a parti
dell'item listato per emulare i tag \texttt{dt} e \texttt{dd}
(\textit{Priorità 2});
\item Non sono presenti citazioni, perciò 3.7 è rispettato;
\end{enumerate}

\subsubsection{Chiarire l'uso di linguaggi naturali}
\begin{enumerate}
\item Non viene mai segnalato il cambio linguaggio, ad esempio \textit{home} o
\textit{user} in \texttt{accesso.html} (\textit{Priorità 1});
\item Non vengono mai segnalate abbreviazioni, ad esempio \textit{LDV} in
qualsiasi pagina di un libro o ``\textit{J.K. Rowling}" in
\texttt{ricerca.html} (\textit{Priorità 3});
\item Viene correttamente identificato il linguaggio del documento, perciò 4.3
è rispettato.
\end{enumerate}

\subsubsection{Creare tabelle che si trasformino in maniera elegante}
\begin{enumerate}
\item Non sono presenti tabelle, perciò 5.1 è rispettato;
\item Non sono presenti tabelle, perciò 5.2 è rispettato;
\item Non sono presenti tabelle, perciò 5.3 è rispettato;
\item Non sono presenti tabelle, perciò 5.4 è rispettato;
\item Non sono presenti tabelle, perciò 5.5 è rispettato;
\item Non sono presenti tabelle, perciò 5.6 è rispettato.
\end{enumerate}

\subsubsection{Assicurarsi che le pagine che danno spazio a nuove tecnologie
si trasformino in maniera elegante}
\begin{enumerate}
\item Il contenuto, seppur presentando errori di markup, è sufficientemente
leggibile e comprensibile senza fogli di stile, perciò 6.1 si può ritenere
rispettato;
\item Non si ritiene che i contenuti dinamici del sito possano essere
modificati una volta inseriti, se non per i commenti in coda ai libri;
tuttavia l'aggiornamento automatico tramite AJAX o altre tecnologie esulava
dagli obiettivi del corso di Tecnologie Web, perciò 6.2 si può ritenere
rispettato;
\item Le pagine perdono gran parte della loro usabilità disattivando
JavaScript, ad esempio ogni form, anzichè avere un input di tipo
\textbf{submit}, gestisce la sottomissione dei dati inseriti via JavaScript
con un input di tipo \textbf{button}, compromettendo normali operazioni che
sarebbero possibili utilizzando nella struttura i tag semanticamente corretti
(\textit{Priorità 1});
\item I gestori degli eventi non dipendono da particolari dispositivi che
utilizzano JavaScript, perciò 6.4 è rispettato;
\item Le pagine che non risultano accessibili disattivando JavaScript non
forniscono alcuna pagina o presentazione alternativa (\textit{Priorità 2}).
\end{enumerate}

\subsubsection{Assicurarsi che l'utente possa tenere sotto controllo i
cambiamenti di contenuto nel corso del tempo}
\begin{enumerate}
\item Nessun componente nella pagina causa fenomeni di sfarfallio, perciò 7.1
è rispettato;
\item Nessun componente nella pagina causa fenomeni di lampeggiamento, perciò
7.2 è rispettato;
\item Nessun componente nella pagina si muove una volta caricato, perciò 7.3 è
rispettato;
\item Le pagine non si aggiornano automaticamente, perciò 7.4 è rispettato;
\item Non vi sono pagine che effettuano auto-reindirizzamento, perciò 7.5 è
rispettato.
\end{enumerate}

\subsubsection{Assicurare l'accessibilità diretta delle interfacce utente
incorporate}
\begin{enumerate}
\item Gli script sono ritenuti accessibili verso i dispositivi che li
supportano, perciò 8.1 è rispettato.
\end{enumerate}

\subsubsection{Progettare per garantire l'indipendenza da dispositivo}
\begin{enumerate}
\item Non sono presenti mappe immagine, perciò 9.1 è rispettato;
\item Ogni elemento è accessibile indipendemente dal dispositivo con cui si
naviga nel sito, perciò 9.2 è rispettato;
\item Gli eventi si riferiscono a gesture o actions astratte o logiche e non
relative a specifici dispositivi, perciò 9.3 è rispettato;
\item Non è specificato un ordine di tabulazione diverso da quello di default,
sebbene potrebbe essere utile in questo sito (\textit{Priorità 3});
\item Non sono definite scorciatoie da tastiera, il che potrebbe essere
positivo o negativo a seconda delle motivazioni che risiedono dietro questa
scelta; 9.5 si può ritenere rispettato.
\end{enumerate}

\subsubsection{Usare soluzioni provvisorie}
\begin{enumerate}
\item La finestra attiva non viene mai cambiata nè vengono generate nuove
finestre nè appaiono finestre a cascata, perciò 10.1 è rispettato;
\item Sebbene le label degli input nei form siano associate in modo erroneo
(come discusso nella sezione \ref{sec:acc-form}), queste sono posizionate
correttamente a livello presentazionale rispetto ai rispettivi input, perciò
10.2 è rispettato;
\item Non ci sono tabelle, perciò 10.3 è rispettato;
\item I placeholder ci sono, ma non sempre funzionano nel modo aspettato (come
discusso nella sezione \ref{sec:acc-form}) (\textit{Priorità 3});
\item Non sono presenti molti link nelle pagine e nessuno di questi è
adiacente a un altro senza altri elementi o tag tra i due link, perciò 10.5 è
rispettato.
\end{enumerate}

\subsubsection{Usare le tecnologie e le raccomandazioni del W3C}
\begin{enumerate}
\item Vengono utilizzate solamente tecnologie W3C (anche se talvolta non
rispettando gli standard), perciò 11.1 è rispettato;
\item Le label dei form sono associate al contenuto degli attributi
\texttt{name} (come detto nella sezione \ref{sec:acc-form}) e non a quello
degli attributi \texttt{id}, come da raccomandazione W3C; essendo pratica non
prevista dall'ente, presumibilmente è anche disapprovata
(\textit{Priorità 2});
\item Non si vedono nè localizzazione nè globalizzazione come prospettive di
tale sito, perciò 11.3 è rispettato;
\item Durante la realizzazione del sito probabilmente non è stato notato che
le pagine non rispettano gli standard o che presentano problemi di
accessibilità, perciò non sono fornite versioni alternative
\textit{accessibili} delle pagine del sito (\textit{Priorità 1}).
\end{enumerate}

\subsubsection{Fornire informazione per la contestualizzazione e
l'orientamento}
\begin{enumerate}
\item Non sono presenti \textbf{frame}, perciò 12.1 è rispettato;
\item Non sono presenti frame, perciò 12.2 è rispettato;
\item I campi delle form sono raggruppati con tag \texttt{fieldset} e
\texttt{legend}, perciò 12.3 è rispettato;
\item Come descritto nella sezione \ref{sec:acc-form}, i tag \texttt{label}
non sono associati agli input correttamente (\textit{Priorità 2}).
\end{enumerate}

\subsubsection{Fornire chiari meccanismi di navigazione}
\begin{enumerate}
\item I link fanno effettivamente capire dove portano, ma mancano gli
attributi \texttt{title} per spiegare meglio cosa riferisce l'ancora, perciò
13.1 si può ritenere rispettato in modo lasco;
\item L'utilizzo dei meta tag è sempre scorretto, poichè i contenuti dei pochi
tag presenti sembrano presi da un template XHTML (ad esempio in
\texttt{about.html} il tag \texttt{description} ha come contenuto
\textit{Inserisci le descrizioni}) e mancano tag importanti, come
\texttt{author} (\textit{Priorità 2});
\item Non è presente nessuna pagina dedicata all'orientamento sul sito, nè
mappa nè un indice ben definito (non è presente nemmeno una \textit{sitemap}
\footnote{\texttt{https://support.google.com/webmasters/answer/183668?hl=it}}
vera e propria), non rispettando di fatto il punto di controllo 13.3
(\textit{Priorità 2});
\item Come descritto nella sezione \ref{sec:user-ui_consistency}, gli elementi
di navigazione non sono consistenti o coerenti nelle varie pagine del sito
(\textit{Priorità 2});
\item Il sito non dispone sempre di una barra di navigazione, ad esempio nella
pagina visualizzata se si fallisce l'accesso (\textit{Priorità 3});
\item Il sito non ha necessità di forkare codice per diversi
\textit{user agent}, perciò 13.6 è ritenuto rispettato;
\item Il sito offre diversi metodi di ricerca (sebbene questa non funzioni
correttamente come descritto nella sezione \ref{sec:schema-org}), perciò 13.7
è rispettato;
\item In alcune parti l'informazione più significativa non è posta all'inizio
di elenchi, come descritto nella sezione \ref{sec:schema-elenchi}
(\textit{Priorità 3});
\item Non sono presenti documenti composti da più pagine, perciò 13.9 è
rispettato;
\item Non sono presenti immagini realizzate con l'\textit{ASCII multi-line}
\footnote{\texttt{https://trendimpulse.wordpress.com/tag/multiple-lines-ascii-art/}}, perciò 13.10 è rispettato.
\end{enumerate}

\subsubsection{Assicurarsi che i documenti siano chiari e semplici}
\begin{enumerate}
\item Anche se il testo non sempre è grammaticalmente corretto e semplice, si
ritiene che la (\textbf{seconda}) scelta sia dovuta al fatto che l'audience è
un pubblico di cultura letteraria medio-alta (essendo il sito di una
biblioteca), perciò 14.1 si può ritenere rispettato;
\item Sono mostrate le immagini delle copertine dei libri, mentre l'audio
potrebbe essere meno significativo per aumentare il valore del contenuto,
perciò 14.2 è rispettato;
\item Lo stile di presentazione è comune a tutte le pagine, perciò 14.3 è
rispettato.
\end{enumerate}

%%%%%% FINE LINEE GUIDA -> QUI SI METTONO TUTTE LE COSE NON SALTATE FUORI PRIMA

\subsection{Errori non risultati dal confronto con le linee guida}

\subsubsection{Form}\label{sec:acc-form}
I form presentano vari errori:
\begin{itemize}
\item I form sono spesso bizzarramente contrassegnati con la classe
\texttt{forms}, come in \texttt{ricerca.html}, rompendo nuovamente la
separazione presentazione-struttura;
\item Non compaiono gli input di submit, rendendo impossibile la richiesta di
azione di un form se JavaScript è disattivato (si ritiene che non siano stati
messi i submit poichè fosse lievemente più complesso gestire l'arresto
nell'invio della richiesta di sottomissione nel caso in cui i dati inseriti
fossero scorretti, rompendo la separazione comportamento-struttura);
\item Per qualche strano motivo gli input presentano attributi \texttt{name} e
\texttt{id} con contenuto differente;
\item Le label sono associate erroneamente, per cui la pressione sull'elemento
individuato dal tag label non risulta nell'acquisizione del focus sull'input
desiderato;
\item Le ricerche di libri per autore, titolo e collana vengono eseguite con
una chiamata REST di tipo POST, impedendo all'utente di salvare come possibile
segnalibro (una \textbf{boa}, nella \textit{metafora della pesca}) la pagina
contenente i risultati della ricerca effettuata;
\item I placeholder non hanno comportamento del tutto aspettato dall'utente,
dal momento che se l'input perde il focus dopo averlo acquisito ed è vuoto,
non ricompare il placeholder.
\end{itemize}

\subsubsection{Separazione tra elementi del sito}
Qui di seguito vengono evidenziate le incongruenze non precedentemente
segnalate, rispetto alla separazione tra i tre elementi che lo compongono,
ovvero \textbf{struttura}, \textbf{presentazione} e \textbf{comportamento}.

\paragraph{Separazione struttura-comportamento} ~\\
In quasi tutte le pagine HTML c'è codice JavaScript, rompendo in modo netto la
separazione struttura-comportamento.

\paragraph{Separazione struttura-presentazione} ~\\
Alcune classi, come \textit{shiftsx} in \texttt{contatti.html} forniscono
informazioni di carattere presentazionale e non semantico.

Nei form è inserito il tag \texttt{br} a scopo presentazionale, poichè serve
solamente per controllare la posizione del fieldset.

Nella pagina \texttt{registrazione.html} il tag \texttt{br} è usato per far
andare a capo il testo.

Nella pagina \texttt{registrazione.html} è associata la classe
\textit{link\_blok} che presuppone che il link sia visualizzato all'interno di
un proprio blocco, ma ciò non avviene. Il nome della classe non ha carattere
semantico ma presentazionale.

Come in \texttt{generi.html}, sono state inserite classi non necessarie per
modificare l'aspetto presentazionale di elementi (le classi
\textit{adattaicona} associate ai link).

\paragraph{Separazione presentazione-comportamento} ~\\
In \texttt{registration\_control.js}, viene assegnato esplicitamente il colore
di elementi tramite JavaScript (con l'attributo \texttt{style}, inserendo CSS
inline).
