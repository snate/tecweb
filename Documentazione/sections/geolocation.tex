\section{Geolocation}
In questo progetto abbiamo inoltre deciso di implementare un modulo che fa uso delle API HTML5 per la geolocalizzazione. L'introduzione di tale caratteristica ha il puro scopo di migliorare la qualità del sito; in assenza di questa, l'utente avrà comunque accesso a tutte le funzionalità di base.
\begin{flushleft}
Grazie al modulo in questione, dopo aver acconsentito a fornire i dati sulla propria posizione e a seconda della pagina in cui si trova, all'utente verranno mostrate diverse informazioni.
\end{flushleft}

\begin{flushleft}
In particolare, spostandosi sulle pagine “mare.html”o “citta.html”oppure “montagna.html” verranno visualizzate:
\begin{itemize}
\item\textit{La Google map a livello stradale avente tanti marker quante sono le località recensite all'interno della sezione appena raggiunta ed il marker che segnala la posizione attuale;
\item Una descrizione recante per ognuna delle località della sezione, la distanza in km dalla posizione attuale.}
\end{itemize}
\end{flushleft}

\begin{flushleft}
In alternativa, spostandosi sulle pagine delle località recensite (es: “praga.html”) verranno visualizzate:
\begin{itemize}
\item\textit{La Google map a livello stradale ad un livello di zoom maggiore rispetto al precedente ed un solo marker posizionato nel centro della città considerata.
\item Una piccola descrizione che informa sulla distanza tra la città considerata e la posizione attuale.}
\end{itemize}
La scelta di utilizzare il servizio Google Maps è dovuta sia dal fatto che le mappe Google sono le più utilizzate dagli utenti\footnote{ Fonte: comScore Mobile Metrix, analisi effettuata negli Stati Uniti a Giugno 2014.
}, sia per la buona documentazione delle API, sia perchè offre la possibilità all'utente di spostarsi, aumentare il dettaglio di zoom oppure passare in modalità StreetView in modo semplice ed immediato.
\end{flushleft}

\subsection{\textbf{L'implementazione del modulo in dettaglio}}

\begin{flushleft}
Il linguaggio di programmazione utilizzato per sviluppare il componente è JavaScript.
Per aiutarci nella creazione del codice è stata inclusa una libreria di utilità chiamata underscore.js.
(\textit{http://underscorejs.org/})
\end{flushleft}
\begin{flushleft}
Vi è un unico script che viene inizializzato al caricamento della pagina ed è valido sia per le pagine di sezione, (es: “mare.html”) sia per le pagine delle località turistiche (es: “praga.html”).
Questo è stato possibile  attraverso una variabile flag che cerca un elemento specifico del documento html esistente solo in determinate pagine dello stesso tipo. In tal modo sappiamo esattamente dove ci troviamo e di conseguenza sappiamo come differenziare il contenuto finale.
\end{flushleft}

\begin{flushleft}
Per ottenere i dati voluti abbiamo dovuto realizzare una base-dati con le informazioni delle località recensite attraverso un oggetto JavaScript così formato:
\end{flushleft}
\begin{verbatim}
   var località = {
       "Parigi": {
	          "name": "Parigi",
	          "loc":"Città",
	          "lat": 48.856614,
	          "lon":  2.3522219
       }
   }
\end{verbatim}
\begin{flushleft}
La funzione principale dalla quale estrapoliamo poi tutti i dati è \textit{getLocation()}.

Invocando questa funzione innanzitutto verifichiamo se il browser supporta la geolocalizzazione e, in caso negativo restituiamo un messaggio di errore, in caso positivo invece viene chiamata la funzione \textit{navigator.geolocation.getCurrentPosition(showPosition}) la quale recupera latitudine e longitudine della posizione attuale.
\end{flushleft}
\begin{flushleft}
La funzione \textit{CalcolaDistanza(mylat,mylon,lat,lon)}, come dice il nome, si occupa di calcolare la distanza fra 2 coordinate geografiche; essa prende come argomenti latitudine e longitudine dei due punti desiderati ed attraverso la formula dell'emisenoverso ne restituisce la distanza in km.
\end{flushleft}

\begin{flushleft}
La funzione \textit{showPosition(position)} ha lo scopo di creare l'interfaccia per i dati ed è infatti la funzione più corposa dello script; essa effettua :
\end{flushleft}
\begin{itemize}
\item L'inizializzazione e successivamente la collocazione  della Google map con i marker nel documento html.
\item Il filtraggio delle sole località da considerare.
\item Il riordinamento per distanza minore dalla posizione attuale delle località nelle pagine di sezione.
\item La creazione e la collocazione degli elementi html che costituiscono le informazioni di distanza.
\end{itemize}

\begin{flushleft}
Lo script è stato testato con i Browsers Google-Chrome ver.40, Firefox ver.35 sia in versione desktop che mobile.
\end{flushleft}



















