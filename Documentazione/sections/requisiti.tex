\section{Individuazione degli utenti e delle loro esigenze}
Il sito non si pone dei vincoli al target di utenza mirato; in particolare si individuano due macro-categorie di utenti di lingua italiana:
\begin{itemize}
\item L'utente che vuole fare una vacanza e desidera avere più informazioni su
questa, conoscendo già la meta desiderata;
\item L'utente che è alla ricerca di informazioni generiche su una località o
di una vacanza senza conoscerne la meta, navigando senza un obiettivo preciso
all'interno del sito.
\end{itemize}
Si è comunque cercato di rendere accessibile le pagine del sito in modo tale
che questo potesse degradare elegantemente in caso di browser vecchi o testuali
e che il sito fornisse supporti a persone svantaggiate sotto il profilo fisico
o psichico.
Per ogni località sono state individuati dei punti di interesse che sono
ritenuti di notevole attrattiva per il target di utenti scelto, cercando di
includere sia un pubblico giovane che uno più adulto. Tuttavia link esterni che
portano a pagine non in lingua italiana sono stati affiancati da un'indicazione
testuale della lingua in cui è scritta la pagina riferita.
La data, tuttavia, è in formato big-endian (americano AAAA-MM-GG) perchè
l'utenza a cui il sito si rivolge è più che in grado di apprendere facilmente
questo formato, se non è addirittura già conosciuto (a differenza di una
possibile soluzione come middle-endian, ovvero AAAA-GG-MM che potrebbe creare
molta confusione nelle categorie interessate).
