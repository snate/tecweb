\section{Front-end}
Gli script introdotti per la parte front-end, interamente sviluppata in
\textbf{JavaScript}, offrono funzionalità che permettono una migliore
esperienza per l'utente.


Nello specifico sono state inserite le seguenti features:
\begin{itemize}
\item un cookie per il salvataggio del nome utente al momento della
pubblicazione di un nuovo commento;
\item il placeholder nella barra di ricerca località;
\item uno script relativo ai commenti utente che si occupa di controllare,
caricare e nascondere i dati inseriti nella form di pubblicazione;
\item un modulo che effettua la geolocalizzazione.
\end{itemize} 
Per ogni caratteristica sopra elencata si è cercato di separare il codice nel
modo più chiaro possibile; a tal fine si è scelto di utilizzare la libreria
require.js (\texttt{http://requirejs.org}) che si occupa essenzialmente di
caricare files situati in percorsi diversi modularizzando quindi i vari
script.

Require.js si è dimostrato un valido \textit{module loader} anche per la sua
estesa compatibilità con i browser (IE 6+, Firefox 2+, Safari 3.2+, Chrome
3+, Opera 10+).

Nello script \texttt{main.js} sono presenti delle variabili globali; tuttavia,
queste vengono inizializzate ed utilizzate solo quando ciò è sia possibile che
necessario (ad esempio una variabile relativa ad un elemento presente solo
nelle pagine di terzo livello non viene inizializzata al caricamento della
homepage).

\subsection{Gestione cookie}
Per immagazzinare persistentemente la username dell'utente abbiamo deciso di
creare un cookie.

Lo scopo di questo è velocizzare la compilazione di un commento; infatti,
l'utilizzatore, avendo pubblicato una prima volta, alla sua seconda troverà
già riempito il campo nome utente.

Nei nostri script sono state implementate una funzione per l'inizializzazione
ed una per il recupero/lettura del cookie appena creato.

In fase di creazione del cookie, oltre alla coppia chiave-valore abbiamo
aggiunto come terzo parametro una data di scadenza impostata a 15 giorni.
La funzione di recupero restituisce il valore del cookie se questo è presente,
successivamente il valore verrà inserito nel documento HTML e l'utente vedrà
la username già compilata.

\subsection{Inserimento placeholder in barra di ricerca}

Poichè il nostro progetto è conforme allo standard XHTML 1.0 Strict ed  il tag
\textit{placeholder} non vi è definito, abbiamo dovuto usare una funzione
JavaScript per riuscire nel nostro intento di includerlo.


Si è quindi deciso di collegare la gestione degli eventi \textit{onBlur()} e
\textit{onFocus()} a due funzioni definite nello script \texttt{main.js}.
Queste due funzioni assicurano un comportamento tale che se la barra di ricerca
non possiede il focus e non contiene testo inserito dall'utente, visualizza un
\textbf{placeholder} ``Nome località" per aiutare l'utente a capire cosa
inserire in quella barra di ricerca, fornendo una classe diversa a questo
testo per enfatizzarla e distinguerla dal testo inserito dall'utente.

Se invece la barra di ricerca acquisisce il focus, vengono aumentate le
dimensioni della stessa, scompare il placeholder (se presente) ma non un
eventuale testo inserito dall'utente. A questo punto l'utente può inserire un
testo, al quale verrà attribuita la classe predefinita prevista per il testo.

Se la barra di ricerca perde il focus, allora questa continuerà a visualizzare
il testo inserito dall'utente oppure verrà nuovamente visualizzato un placeholder (con classe di enfasi), nel caso in cui l'utente non avesse inserito testo.


\subsection{Gestione dei commenti}
Nelle pagine delle località (sezione \ref{sec:struttLoc}) sono stati
inseriti dei pulsanti per visualizzare, nascondere e pubblicare commenti; qui
di seguito viene descritto il modo con cui queste tre operazioni
(visualizzazione, rendere invisibili e pubblicazione dei commenti) sono state
gestite sul lato client di What To Visit.

\subsubsection{Visualizzazione commenti}\label{sec:visComm}
I commenti sono visibili per tutti i browser fuorchè Internet Explorer, nel
cui caso viene mostrato un messaggio di scuse all'utente per la mancanza della
funzionalità.

Questo è dovuto al fatto che i commenti non sono già presenti nella pagina al
momento del caricamento, ma vengono importati da un file XML e quindi
trasformati tramite un foglio di stile XSLT, che trasforma i commenti in liste
prive di ordine i commenti presenti nel file; in particolare:
\begin{itemize}
\item lo script che applicava il foglio di stile XSLT al file XML forniva
degli output parziali e incompleti utilizzando la funzione prevista per
Internet Explorer, mentre la funzione supportata da tutti gli altri browser
forniva un frammento di documento facilmente manipolabile;
\item il risultato fornito dalla funzione prevista per IE non riusciva ad
essere gestito come elemento nodo ma solamente come stringa, mentre con
la funzione supportata da tutti gli altri browser veniva prodotto un elemento
di tipo nodo sul quale era possibile ricercare tra i nodi figli la lista
contenente i commenti relativi solamente alla località della pagina.
\end{itemize}

Se non vi sono commenti da visualizzare viene visualizzata la stringa
``Non vi sono commenti da visualizzare", altrimenti i commenti vengono
visualizzati a gruppi di due ad ogni pressione del pulsante
``Visualizza commenti" fin tanto che ci sono commenti; la divisione di
documento contenente i commenti viene creata ed allocata nel documento solo
alla prima pressione del pulsante ``Visualizza commenti".

I pulsanti ``Visualizza commenti" può essere in tre stati e questi vengono
assegnati via JavaScript tramite delle classi:
\begin{itemize}
\item \texttt{com-runnable}: non sono visualizzati commenti e si può
richiedere la loro visualizzazione (se l'utente sta usando
Internet Explorer o se non vi sono commenti, solo in seguito alla pressione
del pulsante sarà avvisato dell'impossibilità di mostrargli dei commenti sulla
località)
\item \texttt{com-running}: sono attualmente visualizzati alcuni commenti ma è
possibile visualizzarne altri
\item \texttt{com-notrunnable}: non sono presenti ulteriori commenti da
visualizzare
\end{itemize}

\subsubsection{Rendere invisibili i commenti}
Dopo aver visualizzato tutti i commenti di una località o parte di questi,
l'utente ha la possibilità di nasconderli tramite la semplice pressione di un
pulsante.

Ciò è stato realizzato collegando l'evento onclick del pulsante
``Nascondi commenti" ad una funzione \texttt{nascondi\_c()}, che toglie i
commenti presenti nella divisione in cui erano visualizzati e pone il pulsante
``Visualizza commenti" nello stato \texttt{com-runnable} se erano presenti
commenti.

Il pulsante ``Nascondi commenti" può essere in due stati, assegnati anch'essi
attribuendo delle classi al pulsante:
\begin{itemize}
\item \texttt{com-runnable}: sono visualizzati commenti e si offre all'utente
la possibilità di nasconderli
\item \texttt{com-notrunnable}: non sono visualizzati commenti, perciò per non
creare confusione all'utente si segnala che il pulsante è inutilizzabile
(un'eventuale pressione di questo non produrrebbe alcun output visibile)
\end{itemize}


\subsubsection{Pubblicazione dei commenti}
Nel sito è prevista per gli utenti la possibilità di inserire un proprio commento
inerentemente ad una determinata località. A questo scopo, al termine della sezione 
della pagina dedicata alla visualizzazione dei commenti, vi è il pulsante 
``Pubblica il tuo commento" che permette di visualizzare un form per l'inserimento. 
Tale pulsante può trovarsi in due stati:
\begin{itemize}
\item Disattivo: il form non viene visualizzato
\item Attivo: il pulsante è stato attivato dall'utente causando
quindi la visualizzazione del form per l'inserimento.
\end{itemize}

Il form ha inizialmente, quando disattivo, una classe \texttt{class="hidden"} 
che viene gestita in CSS per nasconderlo. Nel momento in cui l'utente attiva il pulsante
``Pubblica il tuo commento" il codice JavaScript si occupa di modificare tale classe
eliminandone il valore. In questo modo il form compare permettendo all'utente
di inserire il proprio commento. 
Il form è costituito da due campi e due pulsanti così caratterizzati:
\begin{itemize}
\item \textbf{Username:} campo in cui viene richiesto uno username di identificazione
eventualmente compilato automanicamente tramite il valore del cookie precedentemente creato
\item \textbf{Testo del commento:} campo in cui l'utente può inserire il testo del proprio commento
\item \textbf{Pubblica:} pulsante la cui attivazione determina l'invio del commento
\item \textbf{Annulla:}  pulsante la cui attivazione determina l'annullamento dell'operazione
\end{itemize}
A ciascuno dei campi sono associate una o più funzioni che si occupano di controllare i dati
in input. E' prevista infatti l'obbligatorietà di inserire un valore per ognuno dei campi 
\textbf{Username} e \textbf{Testo del commento}. Alla perdita del focus dei campi, se il loro contenuto è vuoto, vengono visualizzati i relativi messaggi di errore.
Il pulsante \textbf{Pubblica} si occupa di attivare il codice Perl che esegue l'effettivo
inserimento del commento all'interno dell'apposito file XML. Il pulsante \textbf{Annulla} invece permette all'utente di annullare l'operazione di pubblicazione del proprio commento.
Lo script ad esso associato, dopo aver eliminato il contenuto dei campi, nasconde il form e
annulla l'operazione.

\subsection{Geolocalizzazione}
In questo progetto abbiamo inoltre deciso di implementare un modulo che fa uso
delle API HTML5 per la geolocalizzazione. L'introduzione di tale
caratteristica ha il puro scopo di migliorare la qualità del sito; in assenza
di questa, l'utente avrà comunque accesso a tutte le funzionalità di base.
\begin{flushleft}
Grazie al modulo in questione, dopo aver acconsentito a fornire i dati sulla
propria posizione e a seconda della pagina in cui si trova, all'utente
verranno mostrate diverse informazioni.
\end{flushleft}

\begin{flushleft}
In particolare, spostandosi sulle pagine ``\texttt{mare.html}" o
``\texttt{citta.html}" oppure ``\texttt{montagna.html}" verranno visualizzate:
\begin{itemize}
\item\textit{una mappa a livello stradale avente tanti marker quante sono le località recensite all'interno della sezione appena raggiunta ed il marker che segnala la posizione attuale;
\item una descrizione recante per ognuna delle località della sezione, la distanza in km dalla posizione attuale.}
\end{itemize}
\end{flushleft}
\begin{flushleft}
In alternativa, spostandosi sulle pagine delle località recensite (es: “praga.html”) verranno visualizzate:
\end{flushleft}
\begin{itemize}
\item\textit{una mappa a livello stradale ad un livello di zoom maggiore rispetto al precedente ed un solo marker posizionato nel centro della città considerata;
\item una piccola descrizione che informa sulla distanza tra la città considerata e la posizione attuale.}
\end{itemize}
È stato scelto di utilizzare Google Maps poiché gli utenti hanno una maggiore
familiarità con le mappe offerte da Google\footnote{ Fonte: comScore Mobile
Metrix, analisi effettuata negli Stati Uniti a Giugno 2014.}, per la qualità
della documentazione relativa alle API offerta da Google e perchè l'utente con
queste mappe riesce facilmente a spostarsi, aumentare il dettaglio di zoom
oppure passare in modalità StreetView in modo semplice ed immediato.

\subsubsection{L'implementazione del modulo in dettaglio}

Vi è un unico script che viene inizializzato al caricamento della pagina ed è
valido sia per le pagine di sezione, (es: ``\texttt{mare.html}") sia per le
pagine delle località turistiche (es: ``\texttt{praga.html}").
Questo è stato possibile attraverso la ricerca di un id presente solamente
nelle pagine relative alle categorie, ma assente nelle pagine delle località.
In tal modo lo script è in grado di discriminare se la pagina è di una
categoria o di una località.

\begin{flushleft}
Per ottenere i dati voluti abbiamo dovuto realizzare una base-dati con le informazioni delle località recensite attraverso un oggetto JSON così formato:
\end{flushleft}
\begin{verbatim}
   var località = {
       "Parigi": {
	          "name": "Parigi",
	          "loc":"Città",
	          "lat": 48.856614,
	          "lon":  2.3522219
       },
       //...
   }
\end{verbatim}
\begin{flushleft}
La funzione principale dalla quale estrapoliamo poi tutti i dati è
\textit{getLocation()}.

Invocando questa funzione viene innanzitutto verificato se il browser supporta
la geolocalizzazione, fornendo un messaggio di errore nel caso in cui non la
supportasse; se invece il browser supporta questa funzionalità, viene chiamata
la funzione \textit{navigator.geolocation.getCurrentPosition(showPosition}) la
quale recupera latitudine e longitudine della posizione attuale del
dispositivo utilizzato dall'utente.
\end{flushleft}
\begin{flushleft}
La funzione \textit{CalcolaDistanza(mylat,mylon,lat,lon)}, come dice il nome, si occupa di calcolare la distanza fra 2 coordinate geografiche; essa prende come argomenti latitudine e longitudine dei due punti desiderati ed attraverso la formula dell'emisenoverso ne restituisce la distanza in km.
\end{flushleft}

\begin{flushleft}
La funzione \textit{showPosition(position)} ha lo scopo di creare l'interfaccia per i dati ed è infatti la funzione più corposa dello script; essa effettua :
\end{flushleft}
\begin{itemize}
\item l'inizializzazione e successivamente la collocazione  della mappa con i marker nel documento HTML;
\item il filtraggio delle sole località da considerare;
\item il riordinamento per distanza minore dalla posizione attuale delle località nelle pagine di sezione;
\item la creazione e la collocazione degli elementi html che costituiscono le informazioni di distanza.
\end{itemize}



















