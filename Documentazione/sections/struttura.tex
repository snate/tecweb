\section{Struttura}
Di seguito sono illustrate le modalità di progettazione della struttura delle
pagine all'interno del sito.

\subsection{Parti comuni a tutte le pagine}
Tutte le pagine sono state scritte seguendo lo standard \textit{XHTML 1.0 Strict} e come codifica è stata scelta UTF-8 dal momento che nel sito sono
presenti parole accentate.
Una sezione \textbf{head} ed una sezione \textbf{body} sono state inserite in
tutte le pagine con la stessa struttura.

\subsubsection{Head} %TODO @Graziano @Carlo
Sono presenti i seguenti tag nelle sezioni head di tutte le pagine:
\begin{itemize}
\item \textbf{title:} permette di visualizzare sulla finestra del browser il titolo della pagina visualizzata, dal particolare al generale
\item \textbf{meta title:} indica il titolo della pagina in un eventuale snippet, anch'esso dal particolare al generale
\item \textbf{meta description:} in questo tag viene inserita la breve descrizione della pagina visualizzata in un eventuale snippet
\item \textbf{meta author:} in questo tag sono indicati i componenti del gruppo
\item \textbf{meta keywords:} parole che aiutano un motore di ricerca a trovare la pagina grazie a dei termini di importanza focale
\item \textbf{meta robots:} tag che indica ad un eventuale spider se indicizzare la pagina e se seguire i link da essa uscenti
\item \textbf{meta keywords:} parole che aiutano un motore di ricerca a trovare la pagina grazie a dei termini di importanza focale
\item \textbf{meta reply-to:} indica l’indirizzo di posta elettronica dell’autore del documento
\item \textbf{meta Classification:} tag che serve ad indicare l'argomento trattato dalle pagine del sito
\item \textbf{meta viewport:} elemento orientato all'ottimizzate del sito per dispositivi multipli, indicando al browser come controllare dimensioni e scala della pagina.
\item \textbf{link shortcut icon:} icona visibile a fianco al titolo della scheda nel browser, aiuta a identificare meglio le schede di \textit{What To Visit} se un utente avesse più schede aperte nel suo browser
\item \textbf{link stylesheet:} collegamento ai vari fogli di stile CSS, questo tag è stato utilizzato più volte in quanto abbiamo voluto consentire una diversa visualizzazione del sito in base al dispositivo utilizzato dall'utente.
\end{itemize}

\subsubsection{Body}
Affinchè l'utente si sentisse il meno disorientato possibile all'interno di \textit{What To Visit}, si è cercato di progettare il sito con un layout essenziale e che mettesse in primo piano il contenuto aspettato in tutte le pagine.
Sono presenti questi elementi strutturali nei corpi di tutte le pagine:
\begin{itemize}
\item un header, dove vi è il logo del sito;
\item un'ampia parte centrale, dove vengono visualizzati i contenuti richiesti dall'utente;
\item un footer, dove sono presenti link ed informazioni di poco rilievo e un'indicazione riguardo la validità della pagina.
\end{itemize}

\subsection{Homepage}
Come pagina principale del sito, si è pensato di esporre in primo piano all'utente la scelta delle tre categorie delle località.
A partire da queste l'utente può arrivare nelle pagine delle categorie, dove può trovare le liste delle località presenti in queste.

Dal momento che la homepage è l'unica pagina facilmente riconoscibile data la sua struttura con tre titoli di indirizzamento, la breadcrumb è stata omessa perchè si è assunto che gli utenti riuscissero a dedurre che si trovano nella homepage quando vi sono dentro (anche grazie all'URL).

Per poter comunque fornire collegamenti alle pagine che non sono di contenuto ma che sono significative (Chi Siamo e F.A.Q.), i link a queste sono stati inseriti nell'header della pagina a fianco del logo; in questo modo, anche se sono di importanza secondaria rispetto ai tre pannelli visualizzati nella pagina, rimangono comunque nella parte visibile del sito quando questo viene aperto (\texttt{http://en.wikipedia.org/wiki/Above\_the\_fold\#In\_web\_design}).

Nel footer, oltre alle indicazioni di validità della pagina, sono stati lasciati i link restanti alle pagine che non sono di contenuto.
