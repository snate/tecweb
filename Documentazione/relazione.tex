%%%%%%%%%%%%%%%%%%%%%%%%%%%%%%%%%%%%%%%%%
% Arsclassica Article
% LaTeX Template
% Version 1.1 (10/6/14)
%
% This template has been downloaded from:
% http://www.LaTeXTemplates.com
%
% Original author:
% Lorenzo Pantieri (http://www.lorenzopantieri.net) with extensive modifications by:
% Vel (vel@latextemplates.com)
%
% License:
% CC BY-NC-SA 3.0 (http://creativecommons.org/licenses/by-nc-sa/3.0/)
%
%%%%%%%%%%%%%%%%%%%%%%%%%%%%%%%%%%%%%%%%%

%----------------------------------------------------------------------------------------
%	PACKAGES AND OTHER DOCUMENT CONFIGURATIONS
%----------------------------------------------------------------------------------------

\documentclass[
article,
10pt, % Main document font size
a4paper, % Paper type, use 'letterpaper' for US Letter paper
oneside, % One page layout (no page indentation)
%twoside, % Two page layout (page indentation for binding and different headers)
headinclude,footinclude, % Extra spacing for the header and footer
BCOR5mm, % Binding correction
]{scrartcl}
\usepackage[italian]{babel}
\usepackage[utf8]{inputenc}
\usepackage{atbegshi}% http://ctan.org/pkg/atbegshi
\AtBeginDocument{\AtBeginShipoutNext{\AtBeginShipoutDiscard}}

\hyphenation{Fortran hy-phen-ation} % Specify custom hyphenation points in words with dashes where you would like hyphenation to occur, or alternatively, don't put any dashes in a word to stop hyphenation altogether

%----------------------------------------------------------------------------------------
%	TITLE AND AUTHOR(S)
%----------------------------------------------------------------------------------------

\begin{document}

\begin{center}
\title{What To Visit}
\author{Graziano Grespan 1003760 \\
Carlo Munarini 1050128 \\
Federica Speggiorin 1051031 \\
Sebastiano Valle 1050123}
\end{center} % The article author(s) - author affiliations need to be specified in the AUTHOR AFFILIATIONS block

%\date{} % An optional date to appear under the author(s)

%----------------------------------------------------------------------------------------

%----------------------------------------------------------------------------------------
%	TABLE OF CONTENTS & LISTS OF FIGURES AND TABLES
%----------------------------------------------------------------------------------------

\maketitle % Print the title/author/date block

\setcounter{tocdepth}{2} % Set the depth of the table of contents to show sections and subsections only

\tableofcontents % Print the table of contents

\listoffigures % Print the list of figures

\listoftables % Print the list of tables

\section{Abstract}
Il progetto consiste nella realizzazione di un sito Web che presenti ai
visitatori alcune località turistiche suddivise tra località marittime,
località montane e località cittadine.
Gli utenti hanno anche la possibilità di esprimere le proprie opinioni sulle
località esposte aggiungendo dei commenti.

\subsection{Suddivisione ruoli}
Durante il progetto le attività sono state ripartite nel seguente modo:
\begin{itemize}
\item \textbf{Struttura:} Sebastiano Valle, Federica Speggiorin, Graziano Grespan ed in misura minore Carlo Munarini
\item \textbf{Presentazione:} Carlo Munarini, ed in misura minore gli altri componenti
\item \textbf{Front-end:} Graziano Grespan, Sebastiano Valle e Federica Speggiorin
\item \textbf{Back-end:} Sebastiano Valle
\item \textbf{Accessibilità:} Sebastiano Valle, Federica Speggiorin
\item \textbf{Validazione e testing:} Tutti i componenti
\item \textbf{Relazione:} Tutti i componenti
\end{itemize}

\subsection{Schema organizzativo}
Lo schema organizzativo adottato è di tipo ambiguo: sebbene qualche località potrebbe appartenere a più categorie (ad esempio sia città che mare), è stato deciso di associare ad ogni località un'unica categoria per la sua caratteristica di spicco.
Questa scelta è dovuta al fatto che sono previste tre modalità di interazione con il sito:
\begin{enumerate}
\item  l'utente sa già quale località cerca e con la barra di ricerca può direttamente trovare ciò che gli interessa (tiro perfetto nella metafora della pesca), altrimenti nel caso peggiore ha due categorie da esplorare se non trova subito ciò che cerca in una categoria;
\item l'utente ha un'idea precisa del tipo di vacanza che ricerca, ma si aspetta di aumentare le proprie conoscenze riguardo a delle mete di mare/città/montagna durante l'esplorazione del sito (trappola per aragoste nella metafora della pesca);
\item l'utente non sa ciò che cerca ma ha solamente un'idea vaga di ciò che gli interessa; in questo caso, permettendogli di scegliere subito ciò che gli interessa maggiormente, si può far avvertire nell'utente una sensazione di serendipità nell'esplorare nuove località di cui non conosceva nemmeno l'esistenza.
\end{enumerate}


%%%%%%%%%%%% B====>
%PER NOI: OGNI SEZIONE CHE AGGIUNGETE BASTA METTERLA NELLA CARTELLA sections
%con un certo nome (facciamo finta che sia pippo.tex) E 
%POI AGGIUNGERE QUI SOTTO \include{sections\pippo} COME HO FATTO CON L'ABSTRACT
%%%%%%%%%%%% B====>

%----------------------------------------------------------------------------------------

\newpage % Start the article content on the second page, remove this if you have a longer abstract that goes onto the second page

\end{document}

%%%%%%%%%%% B====>
% Dovete stare attenti ad avere installato texlive-full e texlive-publisher (o nome simile)
% Per compilare, make da terminale nella folder dove è contenuto questo file (ho già fatto io il makefile)
% Prima di committare, make clean e cancellare *.pdf
%%%%%%%%%%% B====>