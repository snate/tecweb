\section{Presentazione}\label{sec:presentazione}
In questa sezione vedremo come abbiamo deciso che l'utente debba visualizzare il nostro sito.
Come anticipato nel punto 3.3.1, ogni pagina può essere visualizzata in maniera diversa a seconda del dispositivo utilizzato, vi illustreremo quindi le scelte che abbiamo attuato nelle diverse pagine (I, II, III Livello) in maniera tale che potessero essere visualizzate nel migliore dei modi.

Essendo il nostro sito orientato ad un targhet non specifico che vuole apprendere informazioni su diverse località, abbiamo deciso di permettere 4 diversi modi di visualizzare la pagina, introducendo quindi 4 fogli di stile CSS:
\begin{itemize}
\item \textbf{Foglio di stile layout.css:} Questo layout viene applicato a tutti i dispositivi che visualizzano il nostro sito con più di 780px di larghezza, abbiamo però deciso di ottimizzare tale layout in maniera che potesse essere accessibile anche a coloro che utilizzano sreen-reader;
\item \textbf{Foglio di stile playout.css:} Una persona che trova informazioni riguardo ad una località potrebbe decidere scaricare o stampare il contenuto della nostra pagina così da poterlo guardare in un secondo tempo, su un PDF od un foglio stampato. Questo stile serve proprio affinchè un utente possa stampare ogni pagina del nostro sito, visualizzando al meglio le informazioni chiave della pagina
\item \textbf{Foglio di stile tlayout.css:} Questo foglio di stile entra in gioco quando la pagina viene visualizzata in dispositivi con larghezza inferiore a 780px. In tal modo il nostro sito offre una visualizzazione pulita anche su Tablet e Cellulari
\item \textbf{Foglio di stile mlayout.css:} Nel caso What To Visit venga visualizzato su un cellulare o in una finestra con larghezza inferiore ai 480px, ecco che verrà chiamato in causa questo quarto foglio di stile, molto simile al precedente, che permette però un eccellente visualizzazione delle pagine di III Livello anche sui cellulari. Il sito è quindi riconosciuto da Google come \textit{Mobile-Friendly}\footnote{Il \textit{Test di compatibilità con dispositivi mobili} è stato eseguito all'indirizzo: https://www.google.com/webmasters/tools/mobile-friendly/}
\end{itemize}

\subsection{I colori}
Prima di iniziare a guardare nel dettaglio l'aspetto grafico di ogni pagina, vogliamo presentare quelli che sono i colori adottati nel nostro sito. In What To Visit infatti gli utenti avranno di fronte a se delle pagine che utilizzano pochi colori, ai quali abbiamo cercato di attribuire un significato:
\begin{itemize}
\item \textbf{Verde scuro (#1D653C):} Con questo colore, simile ad un Verde primavera scuro, abbiamo voluto indicare gli elementi non attivi ma che possono essere attivati (come bottoni per i commenti o i link), gli elementi fissi della pagina (come footer ed header) oppure elementi non attivabili che però caratterizzano una località o un collegamento (come succede per indicare a quale cateogria appartiene la località o per indicare un link che non è stato visitato e porta in una pagina esterna al nostro sito);
\item \textbf{Verde chiaro (#2ECC71):} Con questa via di mezzo tra un verde primavera ed un verde smeraldo, si è deciso di indicare quegli elementi che sono già stati attivati o che potrebbero esserlo poichè puntati dal cursore. Un esempio possono essere i link già visitati (esterni o interni al sito), link puntati dal cursore o ancora, i bottoni per i commenti qualora siano stati attivati o possano portare al cambiamento/essere frutto di un cambiamento della pagina.
\item \textbf{Grigio Scuro (#444444):} Questo grigio scuro viene utilizzato come colore del testo di contenuto e come colore di background per le caselle di testo nel quale l'utente deve per l'appunto inserire informazioni o commenti.
\item \textbf{Bianco (#FFFFFF):} È presente in tutte le pagine in quanto, colore di background del sito e della barra di ricerca presente nella breadcrumb (Vedi punto 3.5), unica eccezione riguardante il grigio scuro come background-color delle caselle di testo.
\item \textbf{Rosso (#E84444):} Utilizzato solamente in due casi, nelle pagine di III Livello, questo colore indica all'utente che qualcosa non va, o che premendo un determinato bottone potrebbe cancellare i dati inseriti in una form.
\item \textbf{Grigio Chiaro (#C9C9C9):} Questo colore viene utilizzato solamente in un caso, quello in cui un bottone, anche se premuto, non cambierebbe la pagina. Questa scelta è dovuta al fatto che si vuol cercare di far capire all'utente che quel bottone è sostanzialmente inutile in quel determinato momento ma che potrebbe essere utilizzato in un secondo momento.
\end{itemize}

\subsection{CSS e Parti comuni a tutte le pagine}
Durante la realizzazione del sito abbiamo cercato di creare un layout semplice, accessibile, utilizzabile ma sopratutto visualizzabile al meglio sul maggior numero possibile di Browser. Il nostro sito limita quindi l'utilizzo del linguaggio CSS3, un liguaggio presente ma allo stesso tempo non fondamentale, che consente quindi un degrado elegante del sito nel caso di mancato supporto a determinate funzioni.

CSS3 è usato in diverse parti del sito, specialmente nelle parti comuni a tutte le pagine, già descritte nella sezione 3.3.2, e specialmente nell'ambito della visualizzazione del sito per dispositivi mobili. Ora però vediamo come vengono visualizzati tali elementi a seconda dei dispositivi utilizzati:
\begin{itemize}
\item L'\texttt{Header} è un elemento presente in tutte le pagine del nostro sito (I, II e III Livello) ma viene visualizzato in maniera differente a seconda che venga utilizzato un foglio di stile piuttosto che un altro. La differenza principale è il modo di mostrare il nome del sito, infatti nella versione \texttt{Screen} e \texttt{Print}, l'utente vedrà sempre comparire in \textit{alto a sinistra} il nome del sito come una scritta, mentre nel layout per \texttt{Dispositivi Mobili}, tramite una tecnica di \textit{image-replacement}\footnote{Abbiamo utilizzato una piccola variante (Text-indent: -10em e non -9999px) del \textit{Phark's Method} - citato anche da Zeldman nel 2003: http://www.zeldman.com/daily/0703b.shtml#au1103} abbiamo deciso di far visualizzare all'utente il logo di What To Visit.
\item Nav
\item Footer
\end{itemize}