\section{Back-end}
Per quanto riguarda la programmazione lato server, il sito \textit{What To
Visit} offre due funzionalità: con la prima l'utente può inserire commenti,
mentre con la seconda si permette all'utente di accedere alla pagina di una
località tramite una searchbar.

\subsection{Inserimento commenti}
Dal momento che il sito offre la funzionalità di visualizzare commenti inseriti
dagli utenti, questi devono essere salvati in uno o più file nel server. Più
precisamente, ogni commento viene salvato in un file XML chiamato
\texttt{commenti.xml} presente nel server.
Questo documento, valido rispetto allo schema \texttt{commenti.xsd}
(progettato seguendo il modello "\textit{Tende alla Veneziana}), contiene
delle liste di commenti divisi per località; ogni commento, a sua volta,
possiede un corpo, un utente che l'ha scritto e una data in cui è stato
scritto (nel formato AAAA-MM-GG come anticipato nella sezione
\ref{sec:requisiti}). Naturalmente, all'inizio una località può non avere
alcun commento oppure, se qualche commento dovesse essere rimosso per cause di
forza maggiore (ad esempio illegalità o perchè viola i diritti d'autore di un
individuo, come specificato nella pagina della Normativa della Privacy), un
tag \texttt{localita} può rimanere senza commenti.

Lo script di inserimento commenti è stato scritto in Perl utilizzando la
tecnologia \textbf{CGI} (Common Gateway Interface), in modo da riuscire
facilmente a ottenere i dati passati con la form per inserire un commento
presente nelle pagine delle località.

Successivamente lo script procede utilizzando la libreria \textbf{LibXML}, con
le cui funzionalità legge il file dei commenti e cerca il nodo
\texttt{localita} in cui inserire il commento con una query XPath.
Una volta trovato, viene creato il frammento XML da aggiungere alla lista dei
commenti già presenti per la stessa località. Una volta aggiunto il commento,
lo script scrive la modifica sul file XML e reindirizza l'utente alla pagina
della località in cui ha inserito il commento.

\subsection{Barra di ricerca}
L'utente durante la navigazione nel sito ha a disposizione anche una barra di
ricerca; tramite questa, può scrivere il nome di una località (case
insensitive in lingua italiana o in lingua originale) e venire reindirizzato
direttamente alla pagina della località ricercata; se la ricerca non dovesse
trovare nessun risultato, l'utente deve visualizzare una pagina che lo informi
che la ricerca non è andata a buon fine.

Per ottenere questo risultato, è stato scritto uno script in Perl utilizzando
la tecnologia \textbf{CGI} per acquisire il parametro \texttt{\$place} dove è
presente la richiesta effettuata dall'utente e, in caso di ricerca fallita,
per creare la pagina da far visualizzare all'utente.

Lo script inizialmente confronta la stringa ricevuta in input con delle
alternative prefissate; se la stringa passata corrisponde ad una di queste,
allora l'utente viene reindirizzato alla località cercata con successo.
In caso contrario, si esegue una stampa di una pagina con la sezione
head composta come quella nelle altre pagine, header come nelle pagine
secondarie, breadcrumb, nav con possibilità di andare nelle pagine delle
categorie, corpo (dove si informa l'utente di cosa è successo) e footer.
