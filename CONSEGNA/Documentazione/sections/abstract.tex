\section{Abstract}
Il progetto consiste nella realizzazione di un sito Web che presenti ai
visitatori alcune località turistiche suddivise tra località marittime,
località montane e località cittadine.
Gli utenti hanno anche la possibilità di esprimere le proprie opinioni sulle
località esposte aggiungendo dei commenti.

\subsection{Suddivisione ruoli}
Durante il progetto le attività sono state ripartite nel seguente modo:
\begin{itemize}
\item \textbf{Struttura:} Sebastiano Valle, Federica Speggiorin, Graziano Grespan e Carlo Munarini
\item \textbf{Presentazione:} Carlo Munarini, ed in misura minore gli altri componenti
\item \textbf{Front-end:} Graziano Grespan, Sebastiano Valle, Federica Speggiorin ed in misura minore Carlo Munarini
\item \textbf{Back-end:} Sebastiano Valle
\item \textbf{Accessibilità:} Sebastiano Valle, Carlo Munarini, Federica Speggiorin
\item \textbf{Validazione e testing:} Tutti i componenti
\item \textbf{Relazione:} Tutti i componenti
\end{itemize}

\subsection{Schema organizzativo}
Lo schema organizzativo adottato è di tipo ambiguo: sebbene qualche località potrebbe appartenere a più categorie (ad esempio sia città che mare), è stato deciso di associare ad ogni località un'unica categoria per la sua caratteristica di spicco.
Questa scelta è dovuta al fatto che sono previste tre modalità di interazione con il sito:
\begin{enumerate}
\item  l'utente sa già quale località cerca e con la barra di ricerca può direttamente trovare ciò che gli interessa (tiro perfetto nella metafora della pesca), altrimenti nel caso peggiore ha due categorie da esplorare se non trova subito ciò che cerca in una categoria;
\item l'utente ha un'idea precisa del tipo di vacanza che ricerca, ma si aspetta di aumentare le proprie conoscenze riguardo a delle mete di mare/città/montagna durante l'esplorazione del sito (trappola per aragoste nella metafora della pesca);
\item l'utente non sa ciò che cerca ma ha solamente un'idea vaga di ciò che gli interessa; in questo caso, permettendogli di scegliere subito ciò che gli interessa maggiormente, si può far avvertire nell'utente una sensazione di serendipità nell'esplorare nuove località di cui non conosceva nemmeno l'esistenza.
\end{enumerate}
